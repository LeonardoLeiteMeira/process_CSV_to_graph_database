\chapter{Introduction}
\label{cha:intro}

Social media is software that is constantly present in a large part of people's lives. Nowadays contents more than 4.7 billion active users, according to \cite{statista}, which is over 93\% of the internet's users, and it tends to grow more until 2027, when this statistic will reach on 5.85 billion active users \cite{statista_util_2027}. So, it presents itself as a market in full growth that contains large companies, such as Meta owner of Facebook, Whatsapp and Instagram and also Microsoft, owner of LinkedIn. More recent emerged TikTok too, company that grown more in 2021 when grown 142\% in market value \cite{growth_tiktok}.

Therefore, the level of influence and social strength that social media has conquered to this day is perceptible. A common question is how these big companies make their money, as most of them are free. The answer to that question is not so easy and a detailed description of this subject is beyond the scope of this work. By \cite{investopedia} the highest percentage of revenue from these companies comes from Adsense. For this a fundamental resource is the large number of users and also a large dataset of their frequently use, with the objective of the Adenses to indicate a large number of people.

Otherwise, the user’s data that are maintained are also important to another functions, for example to make a recommendation of new connections or and content exhibitions. As a result of this, sometimes the data can be skewed when we consider the actives users objectives at which social media, as an example, in the most part of LinkedIn users the media are used in a professional way, meanwhile the TikTok users have an entertainment objective.

Thus, the project purpose is the construction of a system that adds information about different social medias, considering the objectives interests of every person in each networks, to do the most correct and relevant continent recommendations of connections and exhibition. This work will be followed by a mor detailed presentation of the project, and subsequently by your supplementation and presentation of the user interface idealization. Finally, having the results, they are going to be presented and after having the conclusion.


\section{Goals}

The goal of the present work is to propose a model of database to trace the profile of the users in a multisocial aspect, that is, we are researching and crossing information from different social media, taking from this a more complete profile, which can provide more accurate information. This information would be used to propose connection recommendations between different social media, based on the current connection and the interest of each person and each mapped profile.

Another goal of this work is to use the stored dataset to identify the level of influence of a person on a social media, creating a general and tag influence score, serving as parameters for choosing influential people for marketing ads on different networks.


\section{Art State}

At some related papers we may see different approaches to do the recommendations of connections in social medias. Accordingly, in \cite{multi-feature-recommendation} were is purpose a new approach to recommendations were are consider some information about the users, for example the localization, and such information are used as entries to one SVM (Support Vector Machine).

We also see an similar approach in \cite{convolutional-network}, a work in which the users information are utilized in a Convolutional Neural Network and offer as an exits better recommendations based in more complex information of the users.

While in \cite{semantic-analysis-recommendation} the work developed an algorithm of recommendation based in semantic analysis of social medias into a context of learning environments, for this the authors use some semantic web tools which assists at analysis realizations and recommendations. 

To conclude in \cite{prediction-social-network} is purposed a recommendation algorithm based on graphs, where is also done the comparison between the purposed algorithm and the traditional recommendation algorithms. 

A great difference between those works and the present one becomes from a fact that the most part of them utilize the user’s data that comes from a single social media. However much, the implementations or the utilized information vary from work to work, there wasn’t find until then one solution that are based un multiples social networks to make those recommendations. 
